\documentclass[a4paper]{article}
\usepackage{amsmath}
\usepackage{amssymb}
\usepackage{amsfonts}
\usepackage{subcaption}
\usepackage{caption}
\usepackage{graphicx}
\usepackage{fullpage}
\usepackage{textcomp}
\usepackage{listings}
\usepackage[toc,page]{appendix}

% hyperref must be last
\usepackage{hyperref}
\hypersetup{
  backref=true,
  colorlinks=true,
  linkcolor=blue,
  citecolor=blue,
  urlcolor=blue
}

\newcommand\norm[1]{\left\lVert#1\right\rVert}
\title{MANE4280 Assignment 2}
\author{Jianfeng Yan}
\begin{document}
 \maketitle
 
\section*{Question 1}
\subsection*{a. func.m}
\begin{lstlisting}[language=Matlab, 
                   keywordstyle=\color{blue},
                   stringstyle=\color{red},
                   commentstyle=\color{green}]

function [ f ] = func( x )
  if size(x, 1) ~= 3
  fprintf('size of x should be 3');
  exit;
  end
  A = [41, 12, 12; 12, 36, 32; 12, 32, 36];
  f =  besselj(0, dot(x, A*x));
end
\end{lstlisting}
\subsection*{b. calc\_gradient.m}
\begin{lstlisting}[language=Matlab, 
                   keywordstyle=\color{blue},
                   stringstyle=\color{red},
                   commentstyle=\color{green}]
function [ dfdx ] = calc_gradient( func, x )
  dim = size(x, 1);
  dfdx = zeros(dim, 1);
  x0 = x;
  eps = 1.e-12;
  for i = 1 : dim
    x0(i) = x0(i) + eps*1i;
    f0 = func(x0);
    x0(i) = x0(i) - eps*1i;
    dfdx(i) = imag(f0) / eps;
  end
end
\end{lstlisting}

\subsection*{c}
To calculate the gradient, run the following code:
\begin{lstlisting}[language=Matlab, 
                   keywordstyle=\color{blue},
                   stringstyle=\color{red},
                   commentstyle=\color{green}]
x0 = [pi; 1; -exp(1)];
dfdx = calc_gradient(@func, x0)
\end{lstlisting}

\section*{Question 2}
The following function satisfies all the requirements:
\begin{equation*}
  f = (x+y)^2
\end{equation*}
To see this:
\begin{itemize}
  \item Since $f\ge 0$, and $f(0,0) = 0$, $[0,0]$ is the global minimizer, and hence, a local minimizer.
  \item $H = \begin{bmatrix}
    2 & 2 \\ 2 & 2
  \end{bmatrix}$, which is not diagonal. 
  \item One of its eigenvalue of $H$ is $0$, so the Hessian is positive semi-definite, instead of positive definite. Therefore, the second-order sufficient condition is not satisfied.
\end{itemize}

\section*{Question 3}
\subsection*{a}
The following code solves the constrained optimization problem.
\begin{lstlisting}[language=Matlab, 
                    keywordstyle=\color{blue},
                    stringstyle=\color{red},
                    commentstyle=\color{green}]
  param1 = 10; 
  param2 = 1;
  obj = @(x) obj_BB2(x, param1, param2);
  x0 = [0;0];
  optns = optimoptions(@fmincon, ...
                       'SpecifyObjectiveGradient',true, ...
                       'SpecifyConstraintGradient',true, ...
                       'Display', 'iter', 'Algorithm', 'sqp');

  [x, fval] = fmincon(obj, x0, [], [], [], [], [], [], @cnstr_BB2, optns);
\end{lstlisting}
The minimizer is $x^* = \begin{bmatrix}
-1.314362669710740 & -0.209132625285504\end{bmatrix}^T$.

\subsection*{b}
If the design variable does not satisfy the constraints, we can penalize the objective function, that is, make the function value sufficiently large. The penalty can be proportional to $\norm{c(x)}$. Of course there are many choices to construct the penalty. We can choose those that are differential.
\end{document}

